:title: Existentialization
:data-transition-duration: 150

:css: fonts.css
:css: presentation.css


\newenvironment{table}{.. raw:: html

  <table>}{
  </table>
}

\newenvironment{hs}{.. code:: haskell
}{}
\newenvironment{raw}{.. raw:: html

  <pre>}{
  </pre>
}
\newenvironment{error}{.. raw:: html

  <pre class="error">}{    </pre>
}
\newenvironment{custom}{.. raw:: html

  <pre class="highlight code haskell">}{
  </pre>
}
\newcommand{\$}{\begin{verbatim}$\end{verbatim}}
\newcommand{\todo}[2]{#2}
\newcommand{\note}[1]{<span class="new">#1</span>}
\newcommand{\wat}[1]{<span class="wat">#1</span>}
\newcommand{\type}[1]{<span class="type">#1</span>}
\newcommand{\kind}[1]{<span class="kind">#1</span>}
\newcommand{\syn}[2]{<span class="#1">#2</span>}
\newcommand{\pragma}[1]{\{-# LANGUAGE #1 #-\}}
\newcommand{\pragmasyn}[1]{\syn{cm}{\{-# LANGUAGE #1 #-\}}}
\newcommand{\b}[1]{<pre class="highlight haskell code">#1</pre>}

----

:id: title

.. raw:: html

  <script src='https://cdnjs.cloudflare.com/ajax/libs/mathjax/2.7.0/MathJax.js?config=TeX-MML-AM_CHTML'></script>

  <h1>Existentialization</h1>
  <h2>What Is It Good For?</h2>
  <h3>A talk by <span>Sandy Maguire</span></h3>
  <h4>reasonablypolymorphic.com</h4>

----

Slides available.
=================

\begin{raw}
  <h3>reasonablypolymorphic.com/existentialization</h3>
\end{raw}

----

th-dict-discovery

----

Existentialization

- This title is a little misleading
  - it's actually more
  - how can we implement my th-dict-discovery library
  - to get from here to there we're going to need to go through existentialization
  - so it hasn't been false advertising
  - but you're actually going to get MORE THAN YOU WERE BARGAINING FOR
- The problem:
  - We would like to be able to quickcheck properties for instances we write
  - wouldn't it be nice to have tests that AUTOMATICALLY get generated?
  - like, every monad instance i write is actually a monad?? very cool
- there are actually two problems here -- finding the instances, and representing them somehow in a way you can work with them
  - we're only going to focus about the representation problem today
  - partly because i don't have a great story behind the instance finding
  - we'll power through it super quickly at the end just so you dont have blue balls
- existentialization
  - let's say we are javascript programmers and for some reason we want to shove any type we want into a list
  - haskell doesn't let us do this:
    - [1, 2] :: [Int]
    - [True] :: [Bool}
    - [1, True] :: type error, mafucka
  - but WDGAF. even though this isn't actually useful in any way, we want to do it anyway
  - how can we? is haskell suchh a shit language that it can't express this?
  - no! we can do it
  - data Any where
    - Any :: a -> Any
  - here any can be thought of as a container
    - we can stuff any type we want into it, and get back a value of type any
    - this any contains the `a`
    - but we say it is now existential
    - as in, we know it EXISTS, but we don't know what it is
  - so what happens if we try it?
    - f (Any a) = a
    - • Couldn't match expected type ‘t’ with actual type ‘a’
      - because type variable ‘a’ would escape its scope
      - This (rigid, skolem) type variable is bound by
        - a pattern with constructor: Any :: forall a. a -> Any,
  - hmm. let's think about this. what type would this thing have to have?
    - f :: Any -> a
    - f (Any a) = a
  - but recall this the same as saying `forall a. Any -> a`
    - ie "i can give you back any `a` you want"
  - BUT THIS IS NOT TRUE
    - i have a SPECIFIC a inside of my `Any`
    - but i don't know what it is
    - if it's an Int and you ask for a Bool, I can't just give you a bool because i have an int
  - so that's what this means
    - a rigid skolem variable is a type that is existentially quantified
    - you can't leak it out because it doesn't even EXIST outside
  - this kind of solves our problem:
    - [Any 5, Any Bool, Any (show :: Char -> String)] :: [Any]
    - but it's not actaully useful because we can never get any of this data out
    - shit
- follow up
  - as you might guess, this doesn't mean we can't actually do anything useful with the technique
  - just that it requires MORE THINKING
  - let's talk about iterators
    - like in python or whatever
  - we want to be able to produce a series of values
    - and maybe these values depend on some sort of state
    - we don't really care what that state is, so long as we can pull values out of it
  - data Iterator a where
    - Iterator :: s -> (s -> (a, s)) -> Iterator a
  - we can think of an iterator as containing a piece of internal state, along with a function that will use that state to spit out a value and a new state
    - the thing to notice here is that i don't care what the internal state is
    - it doesn't leak out of my type signature
    - so this thing could depend on the weather, or who knows
    - i don't care though
  - we can implement a function that uses an Iterator to spit out as
    - pump :: Iterator a -> (a, Iterator a)
    - pump (Iterator s f) =
      - let (a, s') = f s
      -  in (a, Iterator s' f)
  - this is kind of neat
  - just because we don't know what type is inside of the iterator's state
    - doens't mean that GHC doesn't know that these types are the same
  - so outside of iterator we don't know and can't look at the type
    - but GHC was smart enough to know there is only actually a single type in here
    - even though it doesn't know what it is, it can still reason about it
- more interesting GADT
  - data Dict c where
    - Dict :: c => Dict c
  - notice here that c exists in the type, and so it is not existential. ghc can track it
  - but this is not any old data type
  - we're saying we can only construct Dict c if c is an instance of a typeclass
  - eg Dict (Enum Bool), Dict (Show Int), but not (Dict (Show (Int -> Int))
  - what value does THIS provide us?
  - it means we can pass constraints along as values
    - they're now reified at the value level
  - example
    - maybeShow :: a -> Maybe (Dict (Show a)) -> String
    - maybeShow a (Just Dict) = show a
    - maybeShow _ Nothing = "i don't know how to show that"
  - we only get a proof of Show a inside of the first case
- we can use the same technique to make a more useful any-list
  - data Showable where
    - Showable :: Show a => a -> Showable
  - showList :: [Showable] -> [String]
  - showList = fmap (\(Showable a) -> show a)
- but what we can't do is
  - data Equatable where
    - Equatable :: Eq a => a -> Equatable
  - equate :: Equatable -> Equatable -> Bool
  - equate (Equatable a) (Equatable b) = a == b
- we can't do this because we don't know that the types packed inside of these things are the same
  - implicitly what we have is `(a :: exists. var0)` and `(b :: exists. var1)` and we are trying to say `a == b` which obviously we can't do since they are different types
- eliminators
  - in general, the strategy for doing useful things with existential variables is to introduce eliminators for them
  - if we want to do something useful with a value of unknown type
    - we're going to need to provide a function that can do something FOR ALL types
  - the general form of it is this:
    - eliminate :: SomeExistential -> (forall a. a -> r) -> r
    - the forall a. a bit should be replaced with the definition of the existential
    - for example:
      - eliminate :: Showable -> (forall a. Show a => a -> r) -> r
      - eliminate :: Iterator a -> (forall s. s -> (s -> (a, s)) -> r) -> r
  - the idea is that if can produce some `r` (that i get to choose) from whatever contents are inside the existential
    - then i can produce an r given some existential
- you might be wondering what useful work you can do with an existential value
  - consider this: if the value you're existential over is only an IMPLEMENTATION DETAIL
    - zipkin example
  - or if you don't even care about the existential anyway
- putting it all together
  - we can make a GADT existential over its dict parameters:
  - data SomeDict1 (c :: k -> Constraint) where
    - SomeDict 1 :: c a => Proxy a -> SomeDict1 c
  - we can use this eg `SomeDict1 Show` to get represent that we have a proof of being able to show SOMETHING, even though we don't know what
  - and so we can use the SAME TRICK
    - a list of [SomeDict1 Monad], for example, is a list of Monad instances
    - if someone provided us with such a list, we could use it to generate quicktest checks proving that each instance follows the laws
